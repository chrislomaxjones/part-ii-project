\chapter{Introduction}

This dissertation describes the process of researching, developing and evaluating an implementation of the Multi--Paxos distributed consensus algorithm in OCaml.

\section{Background}

Distributed systems suffer from a number of possible errors and failure modes. Unreliability is present in the network where messages can be delayed, re--ordered and dropped and processes can exhibit faulty behaviour such as stalling and crashing. The result of this is that distributed systems can end up inconsistent states and even unable to make progress. \\

Consensus is the reaching of agreement in the face of such unreliable conditions. Applications such as transaction systems and distributed databases require consensus in order to remain consistent. Consensus algorithms provide a means by which to reach agreement in the face of such unreliability; this is crucial in the design of distributed systems. \\

Paxos is a consensus algorithm first described by Lamport \cite{Lamport:1998:PP:279227.279229} that allows for consensus to be reached under the typical unreliable conditions present in a distributed system. The algorithm relies on processes participating in a voting protocol that tolerates the failure of a minority of processes. \\

Paxos is used internally in large--scale production systems such as Google's Chubby \cite{Burrows:2006:CLS:1298455.1298487} distributed lock service, where it is used to maintain consistency between replicas. Microsoft's Autopilot \cite{autopilot-automatic-data-center-management} system for data centre management also uses Paxos, again to replicate data across machines. The generality of Paxos allows it to be used as an underlying primitive for various distributed systems techniques. State Machine Replication \cite{Schneider:1990:IFS:98163.98167} is a technique whereby any application that behaves like a state machine can be replicated across a number of machines participating in the Paxos protocol. Likewise, atomic broadcast \cite{Rodrigues:2003:ABA:942591.942742} can be implemented with Paxos as an underlying primitive. \\

Over time Paxos has been extended and modified to emphasise different performance trade--offs. Multi--Paxos is the most typically deployed variant which allows for explicit agreement over a sequence of values. Another example, Fast Paxos \cite{fast-paxos}, is a variant that reduces the number of message delays between proposing a value and receiving a response. More recently, Flexible Paxos \cite{DBLP:journals/corr/HowardMS16} is a variant that relaxes the requirement on agreement between participants in the synod protocol. \\

There are also a number of similar algorithms that are not based on Paxos. Viewstamped replication \cite{Oki:1988:VRN:62546.62549} is primarily a protocol for implementing state machine replication that was developed independently of Paxos. Raft \cite{Ongaro:2014:SUC:2643634.2643666} is a modern alternative to Paxos that attempts to be more understandable.

\section{Aims}

The aim of this project was to produce an implementation of the Multi--Paxos variant of the Paxos algorithm. This is the variant that is used most widely in production systems and provides a foundation upon which to perform state machine replication; in this case to replicate a key value store application. \\

From the outset OCaml was to be used as the primary programming language for the implementation of the project. Its powerful type system allows for a number of errors to be rejected at compile time; a quality that is helpful in developing an application with such an emphasis on fault tolerance. OCaml also provides a rich third party eco--system with libraries for crucial components of the system such as concurrency and RPCs. \\

In order to check the fault tolerance of the implementation it is necessary to simulate a distributed system on a simulated network. This simulator is used to check that the algorithm provides consensus in the face of a number of failure modes. It is also used to characterise the performance of the implementation in terms of latency and throughput which is compared to that of another popular open source Multi--Paxos implementation.




